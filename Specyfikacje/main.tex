\documentclass{article}
\usepackage[utf8]{inputenc}

\title{Specyfikacja funkcjonalna}
\author{Patryk Zaniewski}
\date{11.11.2018}

\usepackage{natbib}
\usepackage{graphicx}
\usepackage{polski}

\begin{document}

\maketitle
\tableofcontents
\newpage

\section{Opis projektu}
Jeremi, Stanisław, Tadeusz, Alojzy i Rajmund są spekulantami na rynku walut Forex. Ich celem jest osiągnięcie jak największego zysku na handlu walutami, w myśl starej zasady: "Kup tanio, sprzedaj drogo". W związku z tym potrzebują programu, który rozwiąże ich problem. Do dyspozycji oddają nam plik tekstowy podzielony na dwie części.
\newline\newline
Pierwsza z nich dotyczy aktualnych kursów walut. Każda z linii zbudowana jest w następujący sposób:
\begin{itemize}
    \item ID waluty (np. 0, 1, 2 ...)
    \item Symbol waluty (np. EUR, GBP, USD)
    \item Pełna nazwa (np. euro, dolar amerykański, funt brytyjski)
\end{itemize}
Druga część pliku zawiera aktualne kursy wymiany walut wraz z kosztami tych operacji. Każda z linii zbudowana jest w następujący sposób:
\begin{itemize}
    \item ID kursu (np. 0, 1, 2 ...)
    \item Symbol waluty wejściowej (np. EUR, GBP, USD)
    \item Symbol waluty wyjściowej (np. EUR, GBP, USD)
    \item Aktualny kurs wymiany
    \item Typ opłaty pobieranej przy przewalutowaniu (Stała lub procentowa, pobierana od waluty wyjściowej)
    \item Opłata
\end{itemize}

\section{Funkcjonalność}
Program służy do wyznaczania najkorzystniejszej ścieżki wymiany walut i/lub wyznaczania arbitrażu walutowego. 
\newline\newline Jako argument wejściowy przyjmowany jest plik z danymi dotyczącymi walut. Po załadowaniu pliku program wyświetla informacje w jaki sposób użytkownik powinien wprowadzać dane do konsoli, aby otrzymać wynik oczekiwanej wymiany lub arbitrażu. Po wprowadzeniu danych następuje wyznaczenie najkorzystniejszej ścieżki, bądź arbitrażu.
\newline\newline Jako wynik końcowy użytkownik otrzymuje w zależności od wyboru, maksymalny możliwy zarobek z wymiany walut wraz z historią ich wymian lub ścieżkę arbitrażu dowolnej waluty. W przypadku, gdy wymiana lub osiągnięcie arbitrażu będzie niemożliwe użytkownik otrzyma stosowną informację.

\section{Komunikacja z programem}
Aby poprawnie uruchomić program należy podać jako argument uruchomienia nazwę pliku zawierającego aktualne kursy walut.
\newline\newline
Kolejnym krokiem jest wybranie żądanej operacji. Użytkownik chcąc uzyskać najkorzystniejszą ścieżkę wymiany walut powinien podać trzy argumenty: waluta wejściowa, kwota, waluta wyjściowa. Każdy z tych argumentów powinien być oddzielony białym znakiem. Jeśli jednak użytkownik chce otrzymać arbitraż walutowy powinien podać tylko jeden argument czyli kwotę.
\newline\newline
Działanie programu kończy się poprzez wpisanie polecenia "WYJSCIE" w linii komend.  
\newline\newline



\section{Sytuacje wyjątkowe}
Przewidujemy, że mogą wystąpić sytuacje wyjątkowe. W przypadku poniższych
wystąpi następujące działanie programu:

\begin{enumerate}
\item Użytkownik nie podał argumentu wejściowego czyli nazwy pliku z danymi - program wyświetli wyświetli komunikat o błędzie oraz zakończy działanie. Treść komunikatu: \begin{verbatim}Błąd 01: Nie podano argumentu wejściowego.\end{verbatim}
\item Plik o podanej przez użytkownika nazwie nie istnieje. Treść komunikatu: \begin{verbatim}Błąd 02: Nie znaleziono żądanego pliku\end{verbatim}
\item Nie można było uzyskać dostępu do pliku, ponieważ jest chroniony. Treść komunikatu: \begin{verbatim}Błąd 03: Program nie może uzyskać dostępu do pliku - brak uprawnień\end{verbatim}
\item Wystąpił problem z formatem danych w pliku. Użytkownik otrzyma informację o błędnej linijce oraz możliwe sposoby rozwiązania problemu. Treść komunikatu:\begin{verbatim}Ostrzeżenie 01: Błąd przy odczytywaniu pliku w linii X.
Wpisz "E", aby edytować linię, wpisując "P" pominiesz ją,
natomiast "W" zakończy działanie programu. \end{verbatim}
\item Użytkownik wybrał nieistniejącą walutę jako walutę wejściową. Treść komunikatu:\begin{verbatim}Błąd 04: Waluta wejściowa nie istnieje w pliku.\end{verbatim}
\item Użytkownik wybrał nieistniejącą walutę jako walutę wyjściowa. Treść komunikatu:\begin{verbatim}Błąd 05: Waluta wyjściowa nie istnieje w pliku.\end{verbatim}
\item Użytkownik podał ujemną kwotę wymiany. Treść komunikatu:\begin{verbatim}Błąd 06: Kwota wymiany mniejsza od 0.\end{verbatim}
\end{enumerate}
\end{document}
